%%%%%%%%%%%%%%%%%
% This is an example CV created using altacv.cls (v1.1.5, 1 December 2018) written by
% LianTze Lim (liantze@gmail.com), based on the
% Cv created by BusinessInsider at http://www.businessinsider.my/a-sample-resume-for-marissa-mayer-2016-7/?r=US&IR=T
%
%% It may be distributed and/or modified under the
%% conditions of the LaTeX Project Public License, either version 1.3
%% of this license or (at your option) any later version.
%% The latest version of this license is in
%%    http://www.latex-project.org/lppl.txt
%% and version 1.3 or later is part of all distributions of LaTeX
%% version 2003/12/01 or later.
%%%%%%%%%%%%%%%%

%% If you are using \orcid or academicons
%% icons, make sure you have the academicons
%% option here, and compile with XeLaTeX
%% or LuaLaTeX.
% \documentclass[10pt,a4paper,academicons]{altacv}

%% Use the "normalphoto" option if you want a normal photo instead of cropped to a circle
% \documentclass[10pt,a4paper,normalphoto]{altacv}

\documentclass[10pt,a4paper,ragged2e]{altacv}

%% AltaCV uses the fontawesome and academicon fonts
%% and packages.
%% See texdoc.net/pkg/fontawecome and http://texdoc.net/pkg/academicons for full list of symbols. You MUST compile with XeLaTeX or LuaLaTeX if you want to use academicons.

% Change the page layout if you need to
\geometry{left=1cm,right=9cm,marginparwidth=6.8cm,marginparsep=1.2cm,top=0.25cm,bottom=0.75cm}

% Change the font if you want to, depending on whether
% you're using pdflatex or xelatex/lualatex
\ifxetexorluatex
  % If using xelatex or lualatex:
  \setmainfont{Carlito}
  \usepackage{hyperref}
  \usepackage{xcolor}
\hypersetup{
    colorlinks,
    linkcolor={red!50!black},
    citecolor={blue!50!black},
    urlcolor={blue!80!black}
}
\else
  % If using pdflatex:
  \usepackage[utf8]{inputenc}
  \usepackage[T1]{fontenc}
  \usepackage[default]{lato}
\fi

% Change the colours if you want to
\definecolor{VividPurple}{HTML}{3E0097}
\definecolor{SlateGrey}{HTML}{2E2E2E}
\definecolor{LightGrey}{HTML}{666666}
\colorlet{heading}{VividPurple}
\colorlet{accent}{VividPurple}
\colorlet{emphasis}{SlateGrey}
\colorlet{body}{LightGrey}

% Change the bullets for itemize and rating marker
% for \cvskill if you want to
\renewcommand{\itemmarker}{{\small\textbullet}}
\renewcommand{\ratingmarker}{\faCircle}

%% sample.bib contains your publications
\addbibresource{sample.bib}

\begin{document}
\name{Ibrahim Abdelmonem}
	\tagline{Senior Software Engineer}
% Cropped to square from https://en.wikipedia.org/wiki/Marissa_Mayer#/media/File:Marissa_Mayer_May_2014_(cropped).jpg, CC-BY 2.0
\photo{4cm}{hema3}
\personalinfo{%
  % Not all of these are required!
  % You can add your own with \printinfo{symbol}{detail}
  \email{ibrahim.essam1995@gmail.com}
   \phone{+44 7443775637}
    \location{Bristol, United Kingdom}
      \linkedin{\href{https://www.linkedin.com/in/ibrahimessam}{https://www.linkedin.com/in/ibrahimessam}}
       \github{\href{https://github.com/HemaZ}{github.com/HemaZ}} 
        \homepage{\href{ibrahimessam.com}{ibrahimessam.com}}% I'm just making this up though.
%   \orcid{orcid.org/0000-0000-0000-0000} % Obviously making this up too. If you want to use this field (and also other academicons symbols), add "academicons" option to \documentclass{altacv}
}

%% Make the header extend all the way to the right, if you want.
\begin{fullwidth}
\makecvheader
\end{fullwidth}

%% Depending on your tastes, you may want to make fonts of itemize environments slightly smaller
\AtBeginEnvironment{itemize}{\small}

%% Provide the file name containing the sidebar contents as an optional parameter to \cvsection.
%% You can always just use \marginpar{...} if you do
%% not need to align the top of the contents to any
%% \cvsection title in the "main" bar.
\cvsection[page1sidebar]{Experience}

\cvevent{Software Integration Engineer}{Kudan}{11/2021 - Ongoing}{Bristol, UK}
\begin{itemize}
\item Created and maintained ROS/ROS2 packages for Kudan's SLAM Libraries.
\item Developed the Backend of KudanStudio application using Python web frameworks.
\item Created Python bindings for Kudan's C++ Libraries. 
\item Supported DevOps team in maintaining ROS CI.
\item Developed Kudan's AMR package and the examples launch files.
\item Prepared Robotics demos for conventions like ROSCon and customers. 
\item Continuously evaluated the SLAM algorithms against customer datasets.
\item Supported customer in integrating the ROS packages and tune the SLAM parameters in their existing systems.
\item \code{Key Technologies: SLAM, C++, ROS, Python}.
\end{itemize}

\divider

\cvevent{R\&D Software Engineer}{Avelabs}{09/2018 - 11/2021}{Cairo, Egypt}
Contributed to the development of Avelab's new Acoustic sensing product (AutoHears) by optimizing and porting existing algorithms to C/C++ for deployment on embedded hardware. 
\begin{itemize}
\item Optimized existing Beamforming and DOA algorithms by offloading the slow mathematics computations to TI DSPs.
\item Created benchmarks for the new developed algorithms.
\item Created ROS packages for the product.
\item Collected datasets to validate our algorithms against it.
\end{itemize}
Developer Advocate For Yonohub.com (A cloud-based system for Autonomous Vehicles,
ADAS, and Robotics).
\begin{itemize}
\item  Created tech content for publication as articles, tutorials, and showcase apps to
effectively demonstrate use cases of Yonohub.
\item Developed new Blocks from the state of the art ML/DL and ADAS Algorithms.
\item Configured Hardware for Local Deployment (Nvidia Jetson AGX Xavier, Raspberry Pi).
\item Created AVS Datasets for Yonohub, e.g. KITTI, DeepDrive, ApolloScape and Comma.ai
	\item \code{Key Technologies: Autonomous Vehicles, ROS, Autoware, ML/DL, Cloud, Embedded Boards}.
\end{itemize}
\divider

\cvevent{Motion Planning and Control Engineer (Contractor)}{AeroVect}{11/2020 - 11/2021}{Remotely}
\begin{itemize}
\item Developed and implemented the motion planning and control software stack for The AeroVect Driver.
\item Designed and developed safety and emergency stopping algorithms for The AeroVect Driver to ensure safe operation in all scenarios. 
\item Designed and executed simulations for testing and verification of The AeroVect Driver, ensuring accuracy,robustness and reliabilit of the system.
\item Integrating ROS with the other software components.
	\item \code{Key Technologies ROS, C++,  Control Theory, Autonomous Driving}.

\end{itemize}

\cvsection[page2sidebar]{Experience}
\cvevent{Bachelor Thesis and Internship}{Daimler AG - Mercedes-Benz R\&D}{02/2017 - 08/2017}{Sindelfingen, Germany}
\begin{itemize}
\item Devleoping a Test Robot for Touch Devices Testing. 
\item Hardware (Robot Construction,Kinematics and Touch Devices)
\item Software (CANoe,CAN-bus,Databases and The Test System)
\item Making Tests on The Touch Devices with the Robot to analyze the state
and develop improvements.
\item  Implementing new Algorithms and Data structures for the Robot in MATLAB.
\item Programming a Graphical User Interface for the System
\item \code{Key Technologies: Delta Robots, MATLAB, CANoe}.
\end{itemize}
\divider

\cvsection[]{Projects}
\begin{itemize}
\item \textbf{Edrak. C++ Library for Visual SLAM.}
\href{https://github.com/HemaZ/Edrak}{GitHub}
\item \textbf{Pure pursuit ROS package for path tracking.}
\href{https://github.com/HemaZ/pure_pursuit}{GitHub}
\item \textbf{C++ Implementation of a BlockChain.}
\href{https://github.com/HemaZ/my_blockchain}{GitHub}
\item \textbf{ngrok-ros. ROS package for ngrok.}
\href{https://github.com/HemaZ/ngrok_ros}{GitHub}
\item \textbf{ros2-android, ROS2 package to use android's phone sensors.}
\href{https://github.com/HemaZ/ros2_android}{Github}
\item \textbf{ROSbag2Videos, Extract videos from ROS bags.}
\href{https://github.com/HemaZ/rosbag_2_videos}{Github}
\item \textbf{Teaching an online ROS2 course on Youtube.}
\href{https://www.youtube.com/playlist?list=PLWIFvNcdSDQupEsNLnAnCI6MwTqvGgkC1}{Playlist}
\item \textbf{pclutils a C++ library for working with PointClouds.}
\href{https://github.com/HemaZ/pcl_utils}{Github}
\item \textbf{BaristaBot a robotics simulation package based on ROS and Gazebo.}
\href{https://github.com/HemaZ/BaristaBot}{Github}
\item \textbf{CarSim SFML and ROS based Car Simulator.}
\href{https://github.com/HemaZ/CarSim}{Github}
\item \textbf{Concurrent Traffic Simulation.}
\href{https://github.com/HemaZ/Concurrent-Traffic-Simulation-CPP}{Github}
\item \textbf{Linux System Monitor C++.}
\href{https://github.com/HemaZ/CppND-System-Monitor}{Github}
\item \textbf{Route Planning Project using A* C++.}
\href{https://github.com/HemaZ/CppND-Route-Planning-Project}{Github}
\item \textbf{Unscented Kalman Filter to estimate the state of multiple cars.}
\href{https://github.com/HemaZ/SFND_Unscented_Kalman_Filter}{Github}
\item \textbf{Particles Filter C++ Implementation.}
\href{https://github.com/HemaZ/particles_filter}{Github}
\item \textbf{Time To Collision System (TTC) based on Lidar and Camera.}
\href{https://github.com/HemaZ/SFND_3D_Object_Tracking}{Github}

\item \textbf{PointClouds Obstacles Detection, Segmentation and Clustering}
\href{https://github.com/HemaZ/SFND_Lidar_Obstacle_Detection}{Github}

\item \textbf{Jupyter-ROS (Contributor)  ROS Support for jupyter notebooks}
\href{https://github.com/RoboStack/jupyter-ros}{Github}

\item \textbf{Longitudinal and Lateral Control in CARLA Simulator}
\href{https://github.com/HemaZ/carla-longlat-control}{Github} --- \href{https://www.youtube.com/watch?v=_ONfGpo1h-4}{Video}

\item \textbf{Deep Reinforcement Learning DQN Agent Playing Space Invaders }
\href{https://github.com/HemaZ/Deep-Reinforcement-Learning}{Github} --- \href{https://www.youtube.com/watch?v=yR3SW-NdS-k&t=3s}{Video}

\item \textbf{Road Semantic Segmentation Using Fully Convolutional Network (FCN) }
\href{https://github.com/HemaZ/sem-seg}{Github}

\item \textbf{Building and Simulating TurtleBot using ROS and Raspberry Pi}
\href{https://github.com/HemaZ/AMR_ROS}{Github} --- \href{https://www.youtube.com/watch?v=ThAjbMSuvAo}{Video}
\item \textbf{Optimal LQG Control of Wind Turbine using
Kalman Filter}

\item \textbf{Non-Linear Controller (Feedback Linearization) for 2D Plotter Robot Arm}

\item \textbf{PID Control of Two-Wheeled Self balancing Robot .}
\href{https://www.youtube.com/watch?v=lp5exim_Jro}{Video}

\item \textbf{Yu-Gi-Oh Video Game in Java}
\href{https://github.com/HemaZ/yu-gi-oh-java}{Github} --- \href{https://www.youtube.com/watch?v=wY9EOFwh1F0}{Video}
\end{itemize}

\cvsection[]{Honors \& Awards}
\cvevent{Academic Achievement Full Scholarship}{The German University in Cairo}{2013-2018}{Cairo, Egypt}

\divider

\cvevent{Ranked 7th in Thanwya Amma (High School)}{The Egyptian Ministry of Education}{2013}{Cairo, Egypt}

\divider


%% If the NEXT page doesn't start with a \cvsection but you'd
%% still like to add a sidebar, then use this command on THIS
%% page to add it. The optional argument lets you pull up the
%% sidebar a bit so that it looks aligned with the top of the
%% main column.
% \addnextpagesidebar[-1ex]{page3sidebar}


\end{document}
